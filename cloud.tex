\section{Model for Cloud Platforms}
The iterative and model based approach optimized configuration parameters for resource utilization. As argued in Insight 1, 
resource utilization as an optimziation function is appropriate with the resources saved can be used by other tasks. On SQL-on-Hadoop
machines, this is not true. Moreover, the memory given to a particular container is determined by the Memory/vCPU of the instance type 
and any variation from the ratio is sub-optimial. Therefore a new optimization function focused on cumulative is more appropriate for the cloud.
The optimization function is described in the next nextion.

Next on cloud platforms, it is important recommend instance types along with configuration parameters. We extend Algorithm \ref{jobfit} to determine
the best instance type for a SQL query.

In the last section, we describe experimental results of these additions to the model.

\noindent\subsubsection*{Cumulative Time Optimization Function}
The optimization function is 
$\mathcal{T} = \sum_{i=1}^{n} t_i$ where $n$ is number of containers and $t_i$ is execution time for $i^{th}$ container. 

\noindent\subsubsection*{Algorithm}

\begin{algorithm}
\caption{optimizeCost} \label{cost_optimize}
\begin{algorithmic}[1]
\footnotesize
\REQUIRE $\mathcal{P}$ is the job parameters (defined in \ref{table:job_params} ), Set $\mathcal{I} = {i_1, i_2, i_3, ...} $ set of instance configurations (defined in \ref{table:inst_conf}), Cost Metric Function $\mathcal{M}$
\ENSURE Returns optimized job parameter across all the instance configurations in $\mathcal{I}$.
\RETURN $\min_{\forall i \in \mathcal{I}} \mathcal{M}(fitJob(\mathcal{P}, i), i)$
\end{algorithmic}
\end{algorithm}

\noindent\subsubsection*{Results}

Here we describe the results. 
