\documentclass[10pt, conference, compsocconf]{IEEEtran}

\usepackage{listings}
\usepackage{booktabs} % For formal tables
\usepackage{algorithm}
\usepackage{algorithmic}
\usepackage{enumitem}
\usepackage{amsthm}
\usepackage{multirow}
\usepackage{cite}
\usepackage{graphicx}
\graphicspath{{fig/}}
\DeclareGraphicsExtensions{.jpeg,.png}

\newtheorem{insight}{Insight}
\newcommand{\eat}[1]{}
\newcommand{\PDcomment}[1]{\newline{\bf Prasad: #1}\newline}


\begin{document}
\title{Automatic Tuning of SQL-On-Hadoop Engines on Cloud Platforms}

\author{
	\IEEEauthorblockN{Prasad M. Deshpande}
	\IEEEauthorblockA{Qubole India\\
		Bengalure, Karnataka, India\\
		prasadmd@acm.org}
	\and
	\IEEEauthorblockN{Amogh Margoor }
	\IEEEauthorblockA{Qubole India\\
		Bengalure, Karnataka, India\\
		amoghm@qubole.com}
	\and
	\IEEEauthorblockN{Rajat Venkatesh}
	\IEEEauthorblockA{Qubole India\\
		Bengalure, Karnataka, India\\
		rvenkatesh@qubole.com}
}

\maketitle

\begin{abstract}
More and more companies are running Big Data workloads on cloud platforms. Configuration tuning of these data engines continues to be an essential but difficult undertaking. Cloud platforms further add to this complexity due to elasticity of compute resources and availability of different machine types. Data engineers have to choose the correct machine type as well as the number of machines along with the other configuration options.

In this paper, we address the problem of automatically determining good configuration parameters for SQL workloads on SQL-On-Hadoop Engines like Hive and Spark, with the goal of optimizing resource usage and cost. We chose to focus on SQL workloads because SQL or SQL-like languages continue to be the most popular choice of ETL engineers and analysts. We propose a model based approach that relies on simple rules and insights into SQL Data Engine behavior and evaluate its effectiveness on both synthetic and real workloads.

\eat{
We present two alternatives -- an iterative method based on repeatedly executing queries with different configuration parameters and a model based method relying on simple rules and insights. We studied the effectiveness and practical usefulness of these methods on both synthetic and real workloads.
}
We show that very simple models \eat{of SQL Data Engines} can provide very good recommendations compared to the default configuration. In fact, we found that these were often better than those chosen by experts on real customer workloads. A somewhat surprising result is that more expensive machines with a larger RAM often do not lead to better outcomes in terms of overall query cost and execution times. The principles behind our model based approach are generic and can be adopted for any big data engine. 
\end{abstract}

 \section{Introduction}

There is humongous amount of data being produced and collected around us.
However, such large volumes of data needs to be processed to get meaningful insights.
Traditional Database systems cannot scale to such volumes of data.
Most shared memory based system would not scale for Petabytes of Data. 
Sharding can help but would not be viable option always. 
Most disk based systems on other hands need more expensive hardware as the data volume increases.
This has led to wide adoption of Big Data technologies everywhere and enterprises are moving their 
data to big data platforms. On the other hand SQL has been de-facto language of choice for 
data manipulation and data access. It is very popular data language among analysts, 
software engineers etc to query, manipulate and visualize data. Hence, it was only imminent that 
SQL would continue to be supported on big data platforms too. 
There are quite a few offerings that provide SQL on big data. There are commercial products like HAWQ and many open source products like Apache Hive, SparkSQL, Presto that provides SQL for big data. SQL workloads on big data can be classified broadly into following categories based on their usage:
\begin{description}
  \item[$\bullet$] ETL queries
  \item[$\bullet$] Reporting queries
  \item[$\bullet$] Ad-hoc queries
\end{description}

These technologies support running queries on structured, semi-structured data like JSON, XML etc and unstructured data. Some of the design goals of SQL on big data are following:
\begin{description}
  \item[$\bullet$] Cheap and horizontally scalable
  \item[$\bullet$] High Concurrency
  \item[$\bullet$] Low latency
\end{description}


SQL-on-Big Data has 3 layers which we have identified for optimizing SQL workloads:
\begin{description}
  \item[$\bullet$] Data Model:
  There are many aspects of Data Model that can be tuned to optimize Reporting or Ad-hoc queries.
  Historical queries can be analysed to recommend data formats for heavily used tables, recommend new partitions 
  or columns to sort on when tables are in columnar format. This is beyond the scope of this paper and we will not discuss about them in this work.
  \item[$\bullet$] Execution Engine:
  In this work we have evaluated popular SQL-on-Big Data tools like Apache Hive and SparkSQL. Apache Hive runs on underlying data processing engines like Hadoop MapReduce, Apache Tez or Apache Spark. SparkSQL runs on Spark. These query engines along with underlying data processing engine offer plethora of configuration parameters to tune. This is a challenging and time consuming task even for experts. As we have observed with large installations there can be thousands of queries being fired on daily basis and manual tuning the stack to cater to most of these queries is infeasible. Even the defaults for these configurations based on general rule of thumb can be way off from the optimal configuration for a particular workload. In this work we would present mathematical model using which this process can be automated for the ETL queries that run periodically. Our work would look at one run of these queries and use it to auto tune the configuration parameters.
  \item[$\bullet$] Cluster:
  Performance and the cost of the Big Data Stack is heavily sensitive to the Hardware configuration of the cluster of machines they run on. Especially when these installations are on Cloud there are many machine types available with different configurations. In this work we would be able to recommend machine type for a particular workload among the multiple options provided. 
\end{description}

 \section{Related Work}
\label{sec:relatedwork}
Most relational databases have auto-tuning tools for physical database design. 
Self-tuning Database Systems: A Decade of Progress\cite{Chaudhuri:2007:SDS:1325851.1325856}
 has a good survey of research and tools in this area. Most database systems have 
not put similar attention to determine best values of configuration parameters 
based on workloads. Traditionally effort has focused on specific class of parameters 
(e.g.\cite{Storm:2006:ASM:1182635.1164220} ) or on ranking
critical configuration parameters\cite{DBLP:conf/icde/DebnathLM08}. All these techniques
are based on using the optimizer cost model to simulate database behavior. The disadvantage is
that the cost model may not model the effects of all important parameters.

iTuned\cite{Duan:2009:TDC:1687627.1687767} is one of the first attempts to holistically 
tune configuration parameters in modern database systems. iTuned consists of a planner
which plans experiments and an executor that run experiments to choose the best parameters
and the best values for a specific workload. iTuned is also a good example of a system that
ran experiments on user workloads and queries compared to using models to simulate data engines.
Our preferred approach focuses on SQL-on-Hadoop engine and uses models to eliminate cost of experiments.
 
The popularity of Apache Hadoop ecosystem (Apache Hive, Apache Spark etc) increased interest
in choosing configuration parameters automatically. The developers of these systems exposed
engines parameters, documented them and encouraged users to change them based on their workload.
The main motivation was that the type of workloads and data processed on these systems were varied
and the default values were suboptimal. It was assumed that the administrators have to manipulate
parameters for a successful installation. Similar to tools in database systems, there are two schools
of techniques - model based and execution based.

Starfish\cite{herodotou2011starfish} was one of the first to model hadoop performance. Starfish 
consists of a \textit{profiler}, \textit{what-if} engine and a \textit{cost-based model}\cite{herodotou2011profiling}. 
The \textit{profiler} collects statistics from previous runs of a customer workload 
such as bytes flowing through the system and time taken. The \textit{what-if engine} simulates 
the behavior of the Map Reduce engine and can predict the effect of changing the value 
of a configuration parameter. The \textit{cost based optimizer} uses the \textit{what-if engine}
and recursive random search (RSS) for tuning the parameters for a Hadoop job. The components of
Starfish were extended with a new component - \textit{Elastisizer}\cite{Herodotou:2011:NOS:2038916.2038934}
- to automatically size clusters on cloud platforms. Our approach uses a very simple model. This approach
is practical because we focus on common SQL operations. 

There are multiple approaches \cite{wu2013self} \cite{lama2012aroma} that use machine learning techniques
to cluster jobs based on their profiles. Based on the profile, the most optimal cluster configuration is
used. Aroma \cite{lama2012aroma} optimizes resource allocation and cost of jobs. In
the offline phase, using a training set, the jobs are clustered
(using variants of k-means) according to their respective signatures.
In the online phase, Aroma trains a SVM which makes
accurate and fast prediction of a job's performance for various
configuration parameters and input data sizes. The optimization function in Aroma considers
the machine type, number of machines and unit cost of the machine types with a constraint on
run time of the query. Our approach uses a rule based mathematical model instead of statistical or machine learning approaches. 
 
CherryPick\cite{Li:2014:MMO:2600212.2600229} and Sandeep Kumar et al.\cite{KumarPLPGB16} represent an alternate school of thought.
Mathematical or statistical models for complex data engines is hard to build. Moreover maintenance of these models 
is also hard as these data engines evolve. In this approach, experiments
with alternate values of configuration parameters are are executed instead of simulated in a 
statistical or mathematical model. The obvious disadvantage is that experiments cost money and the main 
goal of these approaches is to make every experiment count. In \cite{KumarPLPGB16}, the authors use SPSA and Bayesian Optimization in \cite{Li:2014:MMO:2600212.2600229} 
to choose an optimal set of experiments to find the best values for a pre-determined set of configuration parameters.
In Section~\ref{section:iter} we tried an iterative execution approach and found it to be impractical in terms of dollar cost. 
At the same time it is hard to manage detailed models of complex engine and therefore
we propose a simple model focused on important SQL operations.
  

 \section{Optimization Methodology}
\label{sec:optmethod}
We now present the optimization functions, parameters and assumptions to tune SQL-on-Hadoop data engines. Even though the model is specific to run time features of SQL-on-Hadoop engines, the principles can be translated to other engines. 

\subsection{Optimization Function}
The optimization function should consider both cost and performance of a workload. In this section, we develop a metric to capture cost and performance of SQL-on-Hadoop engines running on Hadoop2 Yarn Containers. Hadoop divides machines into containers and each container is assigned a fixed amount of memory and CPUs. Workloads are submitted to Hadoop by Hive or Spark as a set of tasks which are scheduled by Hadoop on containers. The first metric we considered measures the total resource utilization. Since memory and CPU are both important resources, the cumulative resource utilization is given by:
\begin{equation}
\label{eqn:totalresource}
\mathcal{R} = \sum_{i=1}^{n} t_i \times m_i
\end{equation}
In the above equation, $n$ is the number of tasks, $t_i$ is the execution time for task $i$ and $m_i$ is the memory allocated to the container on which the task is scheduled. The time factor in the equation tracks the performance of the query. In a cloud deployment, the dollar cost of running the workload becomes an important metric. If the workload under consideration is the only one running on the cluster, then the cumulative time that the workload is scheduled on all the containers is a good proxy for the cost of the query, as given by:
\begin{equation}
\label{eqn:totaltime}
\mathcal{T} = \sum_{i=1}^{n} t_i
\end{equation}
Different machine types on a cloud platform can have different monetary costs depending on the number of cores, amount of memory, storage and network speeds. The actual dollar cost of a workload taking a cumulative time $\mathcal{T}_x$ on a cluster consisting of machines of type $x$ having a rental rate of $r_x$ per core, per unit time is given by:
\begin{equation}
\label{eqn:totalcost}
\mathcal{C} = \mathcal{T}_x \times r_x
\end{equation}


\subsection{Parameters}


\begin{table*}
	\begin{tabular}{ |l|l|l| } 
		\hline
		Parameter & Hive on MR & Spark \\ 
		\hline
		Mapper memory & mapreduce.map.memory.mb & \multirow{2}{*}{spark.executor.memory} \\
		Reducer memory & mapreduce.reduce.memory.mb & \\
		\hline
		\multirow{2}{*}{Mapper parallelism} & \multicolumn{2}{|c|}{mapreduce.input.fileinputformat.split.maxsize} \\
		& \multicolumn{2}{|c|}{mapreduce.input.fileinputformat.split.minsize} \\
		\hline
		Reducer parallelism & hive.exec.reducers.bytes.per.reducer & spark.sql.shuffle.partitions \\
		\hline
		Executor cores & Not applicable & spark.executor.cores \\
		\hline
	\end{tabular}
	\caption{Parameters of the Job to be optimized}
	\label{table:job_params}
\end{table*}

\mycomment{
\begin{table}
\begin{tabular}{ |l|p {4.5 cm}| } 
 \hline
 Parameters & Description \\ 
 \hline
 mapperTime  & Total mapper time in seconds   \\ 
 numOfMapper & Number of map tasks \\ 
 mapperMemory & Container memory for map tasks  \\ 
 splitSize & Input Split Size \\
 mapperInputBytes & Map input in bytes \\
 mapperOutputBytes & Map output in bytes \\
 mapperOutputRecords & Number of Map output records \\
 reducerTime & Total reducer time in seconds \\
 numOfReducer & Number of reduce tasks \\
 bytesPerReducer & Corresponds to Hive parameter \textit{hive.exec.reducers.bytes.per.reducer} \\
 reducerMemory & Container memory for reduce tasks \\
 ioSort & Total amount of buffer memory in mega bytes to be used for sorting. Corresponds to Hadoop parameter \textit{mapreduce.task.io.sort.mb} \\
 spilledMapRecords & Number of records spilled in Map tasks  \\
 \hline
\end{tabular}
\caption{Parameters of the Job to be optimized}
\label{table:job_params}
\end{table}
}

\begin{table}[h]
\begin{tabular}{ |l|p {4.5 cm}| }
 \hline
 Parameters & Description \\ 
 \hline
 nodeMemory  & Total available memory per node for MR job   \\ 
 cpuPerNode & Number of CPUs per node \\ 
 vCpuPerNode & Number of vCPUs per node  \\ 
 \hline
\end{tabular}
\caption{Instance configuration}
\label{table:inst_conf}
\end{table}

\begin{table}[h]
\begin{tabular}{ |p {1.5 cm}|p {3.5 cm}|p {1 cm} | } 
 \hline
 Parameters & Description & Default\\ 
 \hline
 ioSortFrac & Size of ioSort buffer specified as fraction of mapper memory & 0.4 \\
 maxIOSort & Maximum value for \textit{mapreduce.task.io.sort.mb} & 2047 \\
 reducerFrac & Fraction of Reducer memory to be used as buffer & 0.4 \\ 
 \hline
\end{tabular}
\caption{Global Parameters}
\label{table:global_params}
\end{table}

From OtterTune\cite{vanaken}, a tool to auto tune Databases like MySQL and PostgreSQL, it was found that tuning only few knobs can improve performance significantly. This finding carries over to SQL-on-Hadoop engines as well. We chose a few critical parameters to optimize by consulting experts in the domain. The set parameters can be classified as follows:
\begin{enumerate}
	\item[$\bullet$] Job Parameters: These are the parameters that can be set for each job separately, can thus be tuned for each query in the workload. Table \ref{table:job_params} lists these job parameters for Map-reduce and Spark engine.
	\item[$\bullet$] Instance configuration: These parameters are determined by the machine type and are not tunable per query. Table \ref{table:inst_conf} lists the machine type based configuration.
	\item[$\bullet$] Global Parameters: These are some of the global parameters that are neither query nor machine dependent. They can be considered as constants and can be used to tune the algorithm. These are listed in Table \ref{table:global_params} for Map-reduce engines.
\end{enumerate}

%For Map-Reduce engine, the parameters are described in Tables 1, 2 and 3.


 \eat{\section{Iterative Method}
In the first approach, we actually execute each query multiple times with different configuration parameters to determine the optimal set of parameters. For each candidate set of configuration parameters, the query is run and the target metric, such as total resource usage, is measured. By comparing the metrics across different runs with different configuration parameters, we can determine a good set of parameters to use. 

\subsection{Reducing the search space}
The main challenge in such methods is to limit the number of trials, since each execution takes up resources and has a monetory cost associated with it. Earlier approaches based on iterative execution have used various techniques such as noisy gradient~\cite{} to converge to a solution faster. In our method, we make use of domain knowledge and heuristics to reduce the search space. Specifically, we employed the following strategies to reduce the parameter space to be explored.

\noindent\subsubsection*{\bf Parameter reduction: }
The search space is exponential in the number of parameters to be optimized. As described earlier, there are a large number of paramerters that can be set for any hive query. However, based on the experience of tuning a large number of real customer workloads, we can identify a smaller set of parameters that have a relatively larger impact on the performance of sql queries. We restrict the search to these parameters, thus reducing the search space. These parameters have been listed in Table~\ref{}. The search space with a restricted set of parameters is shown in Figure~\ref{fig:searchspace}(a).
\noindent\subsubsection*{\bf Discretization: }
Each parameter, such as memory or partition size, can take a large number of values. However, it can be observed that small changes in the parameters do not have a significant impact. Thus, instead of trying each possible value, it is sufficient to discretize the parameter range and consider only a subset of values for each parameter. These values are placed at a reasonbale distance from each other so as to hvae a significant impact on the query performance. For example, instead of varying memory in units of 1 MB, we can vary it in multiples of 128 MB. The resulting search space is shown in Figure~\ref{fig:searchspace}(b).
\noindent\subsubsection*{\bf Range reduction: }
The range of values for each parameter is further restricted based on domain knowledge about what a good range for that parameter would be.  The knowledge about a good range can be gained by either talking to experts or by looking at some other metrics. For example, for the Hive on MR engine, consider the mapper\_time metric that measures the average time taken by a mapper. If the mapper time is too low, the overhead of starting the mappers is large compared to the actual work done by the mapper. Since the mapper time is inversely related to the number of mappers, the number of mappers need to be reduced. On the other hand, if the mapper time is large, then the job parallelism is restricted and the end to end clock time taken for the query will be high. In this case, more mappers are needed to reduce the work that each mapper has to do. A good acceptable range for this metric could be from 240s till 1800s.  If a set of config params results in mapper\_time beyond the acceptable range, it should not be considered in the search process. For example, mapper\_time is affected by mapreduce.input.fileinputformat.split.maxsize and the correlation is direct, i.e. mapper\_time  increases as we increase mapreduce.input.fileinputformat.split.maxsize.  Thus the split maxsize should be constrained to a range that will lead to a reasonable mapper\_time. In our experiments, we restricted splitsize to between 128 MB and 1 GB. The resulting search space is shown in Figure~\ref{fig:searchspace}(c).
\noindent\subsubsection*{\bf Dimension independence: } 
We make an assumption that the parameters are not correlated to each other. This enables us to optimize each parameter independently of the others. Thus, rather than exploring all the points in the search space, the algorithm explores only one set of values for each parameter as show in Figure~\ref{fig:searchspace}(d). This is a very strong assumption, which may not hold in practice. For example, the mapper memory (mapreduce.map.memory.mb) and the splitsize (mapreduce.input.fileinputformat.split.maxsize) are correlated, since more memory is needed by the mappers as the splitsize increases if spills are to be avoided. Even in this case, the algorithm will find the best value for memory after fixing the splitsize or the best splitsize after fixing the memory. So overall the configuration chosen will be a reasonably good one.
\begin{figure*}[h]
	\includegraphics[width=\linewidth]{fig/searchspace.png}
	%\vspace*{-15pt}
	\caption{Reducing the search space}
	\label{fig:searchspace}
\end{figure*}

\subsection{Algorithm}
The overall algorithm is listed in Figure~\ref{alg:iterativesearch} and is fairly straightforward. It starts with the default value for each configuration parameter (Line 1). It then iterates over the parameters and for each parameter it explores a range of values from low to high, varying it with a minimum step size (Lines 2--6). It runs the query with the chosen parameter values and measures the metric (such as running time or utilization). It finds the value for which the metric is optimized and fixes the value of the parameter to that value before moving on the next parameter (lines 7--12). Finally, it outputs the set of good parameter values $V$ that are discovered in the process.
\renewcommand{\algorithmicrequire}{\textbf{Input:}}
\renewcommand{\algorithmicensure}{\textbf{Output:}}
\renewcommand{\algorithmiccomment}[1]{// #1}
\begin{algorithm}[h]
	\caption{\bf \textit{Iterative Search}}
	\label{alg:iterativesearch}
	\begin{algorithmic}[1]
		%\vspace{1.3em}
		%\small
		\footnotesize
		\REQUIRE Set $\mathcal{P}$ = $\{p_1, p_2 \ldots p_n\}$ of parameters to be determined, the metric $m$ to be optimized, the query $Q$
		\ENSURE The values for parameters in $\mathcal{P}$ that optimize $m$
		\STATE Let $V$ = $\{v_1, v_2, \ldots v_n\}$ = $\{p_1^d, p_2^d, \ldots p_n^d\}$
		\COMMENT {$p_i^l$, $p_i^d$ and $p_i^h$ denote the low value, default value and high value for parameter $p_i$}		
		\FOR {Param $p_i$ in $\mathcal{P}$}
			\STATE Reset $m_{best}$
			\FOR {Value $v$ from $p_i^l$ to $p_i^h$ in steps of $p_i^s$}
				\STATE Replace $v_i$ by $v$ in $V$
				\STATE Run $Q$ with parameter setting $V$ and measure the metric $m$
				\IF {$m$ is better than $m_{best}$}
					\STATE $m_{best} = m$
					\STATE $v_{best} = v$
				\ENDIF
			\ENDFOR
			\STATE Replace $v_i$ by $v_{best}$ in $V$ \COMMENT{Best value for parameter $p_i$ is found and used in further search}			
		\ENDFOR
		\RETURN Output $V$
    \end{algorithmic}
\end{algorithm}
  }
 \section{Insights into Data Engine Behavior}
\label{sec:insights}
In the previous section, we saw that the iterative approach is not scalable as it is expensive to run jobs multiple times. Therefore we set upon an alternative approach of using mathematical models to recommend new settings. In this section, we study the effect of each parameter on the query cost by performing a set of experiments. These experiments provided key insights that will be the basis to develop a simple mathematical model. 
The assumptions made in the iterative approach (Section~\ref{sec:assumptions}) continue to be applicable. \eat{The experiments focus on a few key parameters, the values are discretized, range of values for each parameter is bounded and the dimensions are independent of each other. }



\subsubsection*{Memory and partition size}
We studied the relationship between the container memory and the amount of data processed by the container, i.e. the partition size by performing a set of experiments on the Spark engine. As before, the experiments were run for the TPC DS dataset (scale 1000) and queries on a 4 node AWS cluster with machine type r3.xlarge. We can classify spark tasks into two types based on the data they process. The input mapper tasks process the data from the input tables and the reducer tasks process the output from the previous tasks in the pipeline. The partition size for the input mapper tasks is referred to as the splitsize and is determined by the parameters %$spark.hadoop.mapreduce.input.fileinputformat.
$split.minsize$ and %$spark.hadoop.mapreduce.input.fileinputformat.
$split.maxsize$. The partition size for the reducer tasks is determined by the reducer parallelism, which is controlled by the parameter $spark.sql.shuffle.partitions$. 

In the first set of experiments, we varied the partition size of the mapper tasks by varying the split parameter while holding all other parameters constant. The splitsize was varied from 150 MB to 600 MB. Based on the container memory, the memory available to each executor core was 1386 MB, which was sufficient to hold even the largest split of 600 MB. Hence, there were no spills in any of the experiments. Figure \ref{fig:varysplitsize} shows the variation of the cumulative run time with the splitsize. Overall, the cumulative run time reduces as the splitsize increases. This is because the number of mapper tasks reduces as the splitsize increases. This leads to overall lesser cpu overheads in terms of the number of tasks to be managed. Also, reducer tasks down the pipeline have to pull their data from lesser number of mappers, leading to lesser network overheads. Since there are no spills, there is no major negative effect of larger split sizes. It can be further observed that, though the cumulative time reduces, the overall reduction is not large, with the average being 6.73\%. This is because the total amount of data processed, IO and network traffic is independent of the parallelism, only the overheads vary. 

\begin{figure}[h]
	\includegraphics[width=\linewidth]{fig/varysplitsize.png}
	%\vspace*{-15pt}
	\caption{Varying split size \protect\footnotemark[1]}
	\label{fig:varysplitsize}
\end{figure}

\footnotetext[1]{Bars in chart are in the same order as that of legends}
In the second set of experiments, we varied the partition sizes for the reducer tasks by varying the %$spark.sql.
$shuffle.partitions$ parameter from 50 to 800. This varies the number of reducer partitions and correspondingly the amount of data processed by each reducer task. The original TPC DS queries have many filters on the input tables, due to which the mappers process lots of data and the reducers process very less data. The overall time is thus dominated by the mapper tasks. To make this experiment on the reducer parallelism meaningful, we modified the queries to remove all the filters. After this change, the reducer tasks become a significant part of the overall query and changes in reducer times reflect in the overall query times. Figure~\ref{fig:varyreducers} shows the variations in the cumulative time with the number of partitions. As before, the overall time for the query increases with parallelism, due to increased overheads. The increase is moderate, for e.g., the 16 times increase in parallelism leads to an average slowdown of 47\%. These experiments lead to the next insight:
\begin{insight}
	\label{insight:parallelism}
	The parallelism of a query can be increased without significant adverse effect to the cumulative time. At the same time, it should not be increased indiscriminately since the increased parallelism leads to increased overheads and increases the overall cumulative time.
\end{insight}
The increased overheads are due to managing a larger number of tasks and increase in the number of communicating channels. For MR engine, the overhead of starting a JVM for each task can further add to the overhead.

\begin{figure}[h]
	\includegraphics[width=\linewidth]{fig/varyreducers.png}
	%\vspace*{-15pt}
	\caption{Varying reducer partitions \protect\footnotemark[1]}
	\label{fig:varyreducers}
\end{figure}


In some of the experiments with the reducer parallelism, we could observe the effect of spills on the query time. For example, $q27$, $q34$ and $q47$ cumulative time decreased when parallelism increased from 100 to 200 partitions. Figure~\ref{fig:q46} shows a graph for $q46$, in which the time reduces by almost 30\% when the number of shuffle partitions increases from 100 to 200. This is due to the fact that at 100 partitions, the amount of data processed by the reducer task does not fit in memory and causes a spill. As the number of partitions increases to 200, each partition becomes smaller and fits in memory, thus avoiding a spill. This leads to the following insight:
\begin{insight}
	\label{insight:spill}
	Spills are expensive as each spill leads to an extra write and read of the data. Thus, spills should be avoided at all costs.
\end{insight}
Spills can be avoided by providing adequate memory to each task or by making more fine grained tasks. Since the memory for each container is fixed, we can either reduce the splitsize (for input mapper tasks) or increase the reducer parallelism (for reducer tasks) so that each task processes only data that can fit in its available memory without causing a spill.

\begin{figure}[h]
	\includegraphics[width=\linewidth]{fig/q46.png}
	%\vspace*{-15pt}
	\caption{Effect of spill}
	\label{fig:q46}
\end{figure}


\noindent\subsubsection*{Memory and instance type}
%We saw that the memory of each container should be set according the Memory/vCPU ratio of the machine. 
Cloud platforms provide a variety of machines which differ in various characteristics such as cpu speed, disk speed, network speed, cpu cores and memory. Usually, machines with higher memory/cpu core are more expensive for the same CPU type. To evaluate if it worth paying extra for higher memory/cpu core, we conducted a set of experiments where we varied the memory available per container while keeping all other parameters fixed. The experiments were run on a Spark engine on the TPC DS dataset (scale 1000) and using the TPC DS workload. The average data read by each query was around 20 to 40 GB. The experiments were run on a 4 node cluster on AWS with machine type r3.xlarge. We varied the memory per executor core from 746 MB to 1386 MB in steps of 200 MB. The partition sizes were at the default value of 128 MB, so there is enough memory in all cases to hold the data and there were no spills. In most cases, it was observed that the cumulative cpu time usually decreases as the container memory increases. The decrease is due to lesser pressure on memory and reduced costs of garbage collection. However, in some cases the time also increases with memory. In either case, the variation in cumulative cpu cost with memory was not very significant. Figure~\ref{fig:varymem} shows the variation in the cumulative time with memory. Overall the average speedup was 3.5\% and the maximum speed up was 18\% ($q3$). 

\begin{figure}[h]
	\includegraphics[width=\linewidth]{fig/varymem.png}
	%\vspace*{-15pt}
	\caption{Varying memory per executor core \protect\footnotemark[1]}
	\label{fig:varymem}
\end{figure}

This behavior is due to the way memory is used by the SQL engines. The memory is mainly used to hold the partition processed by the container (as io.sort.mb in MR and to hold the rdd in Spark). Once there is enough memory to hold the partition in memory (thus avoiding spills), any increase beyond that does not have any additional benefit.
As long as tasks do not spill, the total work done in terms of IO, CPU and network traffic is independent of the available memory and the parallelism factor of the tasks. This leads to the following insight:
\begin{insight}
	\label{insight:mem}
	For a query, the memory per task can be decreased safely without performance degradation by correspondingly decreasing the partition size (increasing parallelism) if needed to avoid spills.
\end{insight}
This implies that 
if a job can be tuned to avoid spills on a cheaper instance with same compute but lesser memory than original instance, then it is generally a good idea to move to cheaper instance for saving cost without any performance degradation.

\subsubsection*{Cores per executor}
For Spark engines, there is an additional parameter of cores per executor. Given a certain number of cores per machine, we have a choice of either running many executors with fewer cores per executor (thin executors), or fewer executors with more cores per executor (fat executors). We studied the effect of various choices by varying the $spark.executor.cores$ from 1 to 8 and correspondingly varying $spark.executor.memory$ from 1152MB to 11094MB. The memory per executor core is the same in all cases. Figure~\ref{fig:varycore} shows the variation in the cumulative time with number of cores. It can be observed that fat executors generally provide better performance. On an average, there was a 25\% speedup on using 8 cores per executor compared to 1 core per executor. Fatter executors perform better because of several reasons such as  better memory utilization across cores in a executor, reduced number of replicas of broadcast tables and lesser overheads due to more tasks running in the same executor process. This leads to the following insight specific to Spark engines:
\begin{insight}
	\label{insight:executorcore}
	Use a single fat executor for each node that uses up all the cores on the node rather many thin executors.
\end{insight}

\begin{figure}[h]
	\includegraphics[width=\linewidth]{fig/varycore.png}
	%\vspace*{-15pt}
	\caption{Varying cores per executor \protect\footnotemark[1]}
	\label{fig:varycore}
\end{figure}

\footnotetext[1]{Bars in chart are in the same order as that of legends}
 \section{Model Based Approach}
\label{sec:modelbased}
We now propose a model for SQL-on-Hadoop engines that builds on the insights described in the previous section. We will describe the model for Hive engine on MR. Models for other engines like Spark are very similar, with just the parameters being named differently.  We will show that even though the model is simple, it is a sufficient approximation to generate good recommendations. 

\subsection{Algorithm}
To model the behavior of a query, we need to know the characteristics of the data processed by the query. This includes the input data size and the output data size for each MR stage of the query processing pipeline. One way to estimate this is to use statistics such as sizes of the tables, number of distinct values for each attribute and histograms that will enable us to estimate selectivities of various operators in the query. However, getting accurate data statistics in a big data environments is very often a challenge. Since we are mainly concerned with ETL queries, we can exploit the fact that these queries are run periodically. SQL-on-Hadoop engines collect a lot of metric and configuration information from the jobs that are executed in the system. The overall approach is thus to use the data collected during a run of the query as inputs to our algorithm to recommend good configuration parameters for future runs of the query. The job metrics and parameters used by the algorithm are listed in Table~\ref{table:job_metrics}. The algorithm also takes as input information about the machine instance type and configuration, as listed in Table~\ref{table:inst_conf}. Besides these, there are some global parameters that can be used to tune the algorithm, as listed in Table \ref{table:global_params}.

\eat{
The inputs to the algorithms are:
\begin{enumerate}
    \item[$\bullet$] Job Parameters: These are the key input parameters of the job that effect performance and cost. Table \ref{table:job_params} defines these job parameters.
    \item[$\bullet$] Instance configuration: Table \ref{table:inst_conf} defines the machine configuration.
    \item[$\bullet$] Global Parameters: These are some of the global parameters that can be used to tune the algorithm. These are defined in Table \ref{table:global_params}.
\end{enumerate}
}

\begin{table}
\begin{tabular}{ |l|p {4.5 cm}| } 
 \hline
 Parameters & Description \\ 
 \hline
 mapperTime  & Total mapper time in seconds   \\ 
 numOfMapper & Number of map tasks \\ 
 mapperMemory & Container memory for map tasks  \\ 
 splitSize & Input Split Size \\
 mapperInputBytes & Map input in bytes \\
 mapperOutputBytes & Map output in bytes \\
 mapperOutputRecords & Number of Map output records \\
 reducerTime & Total reducer time in seconds \\
 numOfReducer & Number of reduce tasks \\
 bytesPerReducer & Corresponds to Hive parameter \textit{hive.exec.reducers.bytes.per.reducer} \\
 reducerMemory & Container memory for reduce tasks \\
 ioSort & Total amount of buffer memory in mega bytes to be used for sorting. Corresponds to Hadoop parameter \textit{mapreduce.task.io.sort.mb} \\
% spilledMapRecords & Number of records spilled in Map tasks  \\
 spilledMapBytes & Bytes spilled in Map tasks  \\
 spilledRedBytes & Bytes spilled in Reduce tasks  \\
 \hline
\end{tabular}
\caption{Job metrics and parameters}
\label{table:job_metrics}
\end{table}

\begin{table}[h]
\begin{tabular}{ |l|p {4.5 cm}| }
 \hline
 Parameters & Description \\ 
 \hline
 nodeMemory  & Total available memory per node for MR job   \\ 
 cpuPerNode & Number of CPUs per node \\ 
 vCpuPerNode & Number of vCPUs per node  \\ 
 \hline
\end{tabular}
\caption{Instance configuration}
\label{table:inst_conf}
\end{table}

\begin{table}[h]
\begin{tabular}{ |p {1.8 cm}|p {3.5 cm}|p {1 cm} | } 
 \hline
 Parameters & Description & Default\\ 
 \hline
 ioSortFrac & Size of ioSort buffer specified as fraction of mapper memory & 0.4 \\
 maxIOSort & Maximum value for \textit{mapreduce.task.io.sort.mb} & 2047 \\
 reducerFrac & Fraction of Reducer memory to be used as buffer & 0.4 \\ 
 minMapTime & Minimum time that a mapper should take & 60s \\
 minRedTime & Minimum time that a reducer should take & 60s \\ 
 \hline
\end{tabular}
\caption{Global Parameters}
\label{table:global_params}
\end{table}


Algorithm \ref{alg:optres} optimizes cumulative resource utilization of a SQL query. It can be seen from Equation~\ref{eqn:totalresource} that resource utilization can be reduced by decreasing either the memory usage or time taken for processing. The algorithm thus aims to reduce the memory usage without increasing the time. Insight~\ref{insight:mem} indicates that memory usage can reduced without adverse effect by increasing parallelism if necessary. Insight~\ref{insight:parallelism} further says that parallelism should not be increased indiscriminately. Thus the algorithm maintains a lower bound on the mapper and reducer time ($minMapTime$ and $minRedTime$). It first computes the rate of processing of the mappers (line 1), based on the time metric of the previous run and uses it to compute the splitSize, such that each mapper takes at least $minMapTime$ (line 2). It then computes the $ioSort$, such that the output of each mapper fits in that buffer to avoid spills as per Insight~\ref{insight:spill} (line 3--4). The output size of a mapper is estimated using metrics from the previous run (line 3). The mapper memory is computed from the $ioSort$, using the global parameter $ioSortFrac$ (line 5). Similar computations are done to determine the reducer parallelism ($bytesPerReducer$) and $reducerMemory$ (lines 6--9). Finally, the expected resource usage based on the new parameters is computed (line 10--12).

\renewcommand{\algorithmicrequire}{\textbf{Input:}}
\renewcommand{\algorithmicensure}{\textbf{Output:}}
\renewcommand{\algorithmiccomment}[1]{// #1}
\begin{algorithm}
	\caption{OptResource}\label{alg:optres}
	\begin{algorithmic}[1]
		\footnotesize
		\REQUIRE  $\mathcal{P}$ is the job metrics and parameters (defined in table \ref{table:job_params}) from one run, $I$  is instance configuration (defined in Table \ref{table:inst_conf}) on which $\mathcal{P}$ is collected, $\mathcal{G}$ is the global parameters defined in Table \ref{table:global_params}
		\ENSURE New job parameters $\mathcal{P}_{new}$ and the expected cumulative resource usage $expectedResUsage$ after optimization.
		\STATE $mapTimePerByte \gets mapperTime/(mapperInputBytes + spilledMapBytes)$
		\STATE $\mathcal{P}_{new}.splitsize \gets minMapTime/mapTimePerByte$
		\STATE $mapOutPerSplit \gets \mathcal{P}_{new}.splitsize \times (mapperOutputBytes/mapperInputBytes)$
		\STATE $\mathcal{P}_{new}.ioSort \gets mapOutPerSplit$
		\STATE $\mathcal{P}_{new}.mapperMemory \gets \mathcal{P}_{new}.ioSort / ioSortFrac$
		\STATE $redTimePerByte \gets reducerTime/(mapperOutputBytes + spilledRedBytes)$
		\STATE $dataPerRed \gets minRedTime/redTimePerByte$
		\STATE $\mathcal{P}_{new}.bytesPerReducer \gets dataPerRed \times (mapperInputBytes/mapperOutputBytes)$
		\STATE $\mathcal{P}_{new}.reducerMemory \gets dataPerRed / reducerFrac$
		\STATE $numMappers \gets mapperInputBytes/\mathcal{P}_{new}.splitsize$
		\STATE $numReducers \gets mapperInputBytes/\mathcal{P}_{new}.bytesPerReducer$
		\STATE $expectedResUsage \gets numMappers \times minMapTime \times \mathcal{P}_{new}.mapperMemory  + numReducers \times minRedTime \times \mathcal{P}_{new}.reducerMemory$
		\STATE \RETURN $\mathcal{P}_{new}, expectedResUsage$
	\end{algorithmic}
\end{algorithm}



%\begin{algorithm}
%\caption{checkFit} \label{checkfit}
%\begin{algorithmic}[1]
%\footnotesize
%\REQUIRE $\mathcal{P}$ is the job parameters (defined in \ref{table:job_params} ), $\mathcal{I}$  is instance configuration (defined in \ref{table:inst_conf})
%\ENSURE Returns \textit{true} if $\mathcal{P}$ can fit into $\mathcal{I}$, otherwise \textit{false}.
%
%\STATE newMemPerCore $\gets \mathcal{I}.nodeMemory / \mathcal{I}.vCpuPerNode$
%\IF {newMemPerCore $> \mathcal{P}.mapperMemory$ and newMemPerCore $> \mathcal{P}.reducerMemory$}
%\RETURN \textit{true}
%\ELSE
%\RETURN \textit{false}
%\ENDIF
%\end{algorithmic}
%\end{algorithm}

\subsection{Results}
We evaluated the effectiveness of $OptResource$ method by running experiments on real workloads. The experiments were carried out for a HIVE on MR engine for a workload consisting of 4 real customer queries. 
Figure \ref{fig:modelbasedresult} shows the benefit predicted by our model and the actual observed benefit for these queries. The results show that the algorithm leads to significant savings in the cumulative resource usage cost, ranging from 70\% for $q3$ to 90\% for $q2$. Further, the actual savings closely match the predicted savings indicating that the model is reasonably accurate.

\begin{figure}[h]
	\includegraphics[width=\linewidth]{chart.png}
	%\vspace*{-15pt}
	\caption{Model Based Result: Predicted reduction in cost versus Actual reduction in cost}
	\label{fig:modelbasedresult}
\end{figure}

 \section{Model for Cloud Platforms}
\label{sec:modelcloud}
The iterative and model based approach optimized configuration parameters for resource utilization. As argued in Insight 1, 
resource utilization as an optimization function is appropriate with the resources saved can be used by other tasks. On SQL-on-Hadoop
machines, this is not true. Moreover, the memory given to a particular container is determined by the Memory/vCPU of the instance type 
and any variation from the ratio is sub-optimal. Therefore a new optimization function focused on cumulative is more appropriate for the cloud.
The optimization function is described in the next section.

On cloud platforms, it is important to recommend instance types along with configuration parameters. We extend Algorithm \ref{jobfit} to determine
the best instance type for a SQL query.

In the last section, we describe experimental results of these additions to the model.

\noindent\subsubsection*{Cumulative Time Optimization Function}
\label{subsubsec:cumulative_time}
The optimization function is 
$\mathcal{T} = \sum_{i=1}^{n} t_i$ where $n$ is number of containers and $t_i$ is execution time for $i^{th}$ container. 

\noindent\subsubsection*{Algorithm}

\begin{algorithm}
\caption{optimizeCost} \label{cost_optimize}
\begin{algorithmic}[1]
\footnotesize
\REQUIRE $\mathcal{P}$ is the job parameters (defined in \ref{table:job_params} ), Set $\mathcal{I} = {i_1, i_2, i_3, ...} $ set of instance configurations (defined in \ref{table:inst_conf}), Cost Metric Function $\mathcal{M}$
\ENSURE Returns optimized job parameter across all the instance configurations in $\mathcal{I}$.
\RETURN $\min_{\forall i \in \mathcal{I}} \mathcal{M}(fitJob(\mathcal{P}, i), i)$
\end{algorithmic}
\end{algorithm}

\noindent\subsubsection*{Results}

We evaluated new cost metric defined for cloud on our customer workload on SparkSQL. Note that we have run our experiments in our customer environment against their real world data set. Due to which we could not use the same queries used in earlier evaluation for previous cost metric. We ran our experiments on Amazon Web Services. Figure \ref{fig:modelbaseddollarresult} shows the reduction percent on absolute dollar cost of running 3 customer queries. Dollar cost is determined by the cost of running a particular AWS instance for cumulative time of running query as defined in Section \ref{subsubsec:cumulative_time}.

\begin{figure}[h]
	\includegraphics[width=\linewidth]{ModelExp.png}
	%\vspace*{-15pt}
	\caption{Model Based Result: Reduction percent in absolute dollar cost}
	\label{fig:modelbaseddollarresult}
\end{figure}

 \section{Conclusions}
\label{sec:conclusions}
In this paper, we have addressed the problem of determining good configuration parameters for SQL workloads on big data platforms, specifically SQL-On-Hadoop engines like Hive on MR and Spark SQL. We started with an iterative algorithm that executes each query multiple times with different configuration parameters. Though this method was able to discover good configurations within a small number (around 10-15) of iterations, we found that it was not feasible for real customers due to the large dollar cost of executing each query multiple times. We then focused on a model based approach and developed models for Hive and Spark SQL based on some insights gained through running many experiments. We further extended the model for cloud platforms that require recommending instance types for the cluster in addition to the other parameters. Our results show that the models were able to provide very good recommendations compared to the default and expert chosen configurations on real customer workloads. 

It may sound surprising that very simple models proved to be very effective in practice. This was possible since we chose to focus on SQL-On-Hadoop workloads rather than generic MR or Spark workloads. This enabled us to reason about the query execution and develop the insights. Future work includes extending the models for other engines like Tez and Presto, exploring more configuration parameters and applying the models on many more real workloads by integrating it with the Qubole product. 

\bibliographystyle{IEEEtran}
\bibliography{sample-bibliography} 

\end{document}
