\section{Introduction}

Companies of all sizes are choosing to host their compute infrastructure on cloud platforms like AWS, Micrsoft Azure and Google Cloud Platform. 
Administration best practices has also carried over from the data center. With experience, engineers are building for the unique properties
of cloud platforms especially elasticity. 

Big Data Workloads are part of this trend. Big data workloads are run on the same technologies as in the data center. As Big Data infrastructure
companies have matured, they are also building features to use elasticity in cloud platforms effectively. For example, Qubole provides Auto-scaling 
and Heterogenous clusters are in Apache Hadoop, Apache Spark and PrestoDb clusters. 

However, administration and capacity planning have not yet caught up with elasticity in cloud platforms. Organizations continue to hire experts but the 
complexity of the big data technologies and workloads combined with elasticity is beyond abilities of humans. There are automated tools but they are 
designed for static clusters. The tools do not consider that more machines or different machines types can be used. Moreover all the tools optimize
for performance only (since cost is fixed after provisioning hardware in the data center) while in cloud deployments cost is also equally important. 

In this paper, we propose a model based approach that uses rules and heuristics to optimize performance and cost. Rules for performance optimization
use data statistics from the catalog or previous runs. Additionally the rules use the machine type to factor in the resources available for a workload.
The model optimizes cost by generating expected run types for all machine types and choosing the best combination of run time and price per machine.

Another important factor is the focus on SQL Workloads. Prior attempts at model based approach required models to simulate every aspect of a 
data engine. The advantage is that the approach is generic but the disadvantage is that it is complex and brittle. Prior research and our results
show that a simple model is good enough for SQL Workloads. We compare the configuration values generated by the model with those determined by experts.
Since experts did not choose optimal parameters, we determined optimal parameters using an iterative approach and compare with it as well.

The rest of the paper is organized as follows: Section 2. explores related work. Section 3. describes an Iterative approach that provides optimal 
configuration with higher confidence. Section 4. describes the Model-Based approach. Section 5. is the Experimental Results.
